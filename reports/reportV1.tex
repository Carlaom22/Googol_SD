\documentclass{article}
\usepackage{graphicx} 
\usepackage{geometry}
\usepackage{float}
\usepackage[utf8]{inputenc}
\usepackage[T1]{fontenc}
\usepackage{titlesec}
\usepackage{hyperref}
\usepackage{tabularx}
\usepackage{enumitem}
\usepackage{fancyhdr}
\usepackage{setspace}
\geometry{a4paper, portrait, margin=1in} 
\setstretch{1.25}
\pagestyle{fancy}
\fancyhf{}
\rhead{Sistemas Distribuídos}
\lhead{Meta 1 - Projeto}
\rfoot{\thepage}

\titleformat{\section}{\normalfont\Large\bfseries}{\thesection.}{1em}{}
\titleformat{\subsection}{\normalfont\large\bfseries}{\thesubsection.}{1em}{}

\begin{document}

\begin{titlepage}
    \centering
    \vspace*{1cm}

    \includegraphics[width=0.4\textwidth]{imagens/dei_thumb.png}

    \vspace{1.5cm}
    {\LARGE \textbf{Meta 1 - Relatório Técnico} \par}
    \vspace{0.5cm}
    \vspace{2.5cm}
    \textbf{Licenciatura em Engenharia Informática} \\
    \textbf{Sistemas Distribuídos}

    \vspace{3cm}
    \begin{tabular}{ll}
        \textbf{Carlos Soares} & 2020230124, uc2020230124@student.uc.pt \\
        \textbf{Miguel Machado} & 2020230124, uc2020230124@student.uc.pt 
    \end{tabular}

    \vfill
    {\large \today \par}
\end{titlepage}

\newpage

\section{Introdução}
Este relatório descreve o desenvolvimento de um sistema distribuído de indexação e pesquisa de páginas web, utilizando Java RMI. O projeto visa aplicar os conceitos de tolerância a falhas, replicação, concorrência e modularidade num ambiente de redes e sistemas distribuídos.

\newpage

\section{Objetivos do Projeto}
\begin{itemize}
    \item Permitir ao utilizador pesquisar termos e obter páginas ordenadas por relevância;
    \item Indexar conteúdos web recursivamente, a partir de links inseridos;
    \item Consultar backlinks e estatísticas (termos mais pesquisados, tempo médio);
    \item Assegurar tolerância a falhas e balanceamento entre múltiplos servidores (barrels);
    \item Permitir execução paralela de crawlers;
    \item Armazenar dados persistentemente mesmo após falhas ou reinícios.
\end{itemize}

\newpage

\section{Componentes do Sistema}
O sistema é composto pelos seguintes módulos:

\begin{itemize}
    \item \textbf{Cliente (SearchClient)}: Interface textual onde o utilizador pode:
    \begin{itemize}
        \item Pesquisar termos;
        \item Consultar estatísticas e backlinks;
        \item Adicionar links à fila central de indexação.
    \end{itemize}

    \item \textbf{Gateway (SearchGateway)}: Encaminha pedidos do cliente para barrels ativos, com tolerância a falhas e balanceamento.

    \item \textbf{Barrels (IndexStorageBarrel)}: Servidores que armazenam os índices invertidos e processam pesquisas. Operam com dados replicados e oferecem persistência via ficheiros.

    \item \textbf{Fila Central (CentralURLQueue)}: Interface RMI responsável por armazenar e fornecer links aos crawlers.

    \item \textbf{Crawler (WebCrawler)}: Consome links da fila, faz scraping da página com JSoup, extrai texto e links e envia para um barrel.

    \item \textbf{LinkAdder}: Aplicação de linha de comando usada para adicionar links à fila.
\end{itemize}

\newpage

\section{Funcionamento Geral}

\subsection{Fluxo de Indexação}
\begin{enumerate}
    \item Utilizador insere um link pelo menu (opção 4);
    \item O link é adicionado à \textit{CentralURLQueue};
    \item Um \textit{WebCrawler} ativo consome o link da fila;
    \item O crawler extrai o texto e os links da página com a biblioteca JSoup;
    \item A informação é enviada para um barrel disponível;
    \item O barrel atualiza os índices e armazena backlinks.
\end{enumerate}

\subsection{Fluxo de Pesquisa}
\begin{enumerate}
    \item Utilizador pesquisa um termo pelo cliente;
    \item A pesquisa é encaminhada à \textit{SearchGateway};
    \item A gateway escolhe um barrel disponível e envia o pedido;
    \item O barrel devolve os resultados ordenados por backlinks;
    \item O cliente apresenta os resultados 10 por página.
\end{enumerate}

\subsection{Outras Funcionalidades}
\begin{itemize}
    \item Estatísticas de uso: top 10 termos e tempo médio de resposta;
    \item Consulta de backlinks de uma URL;
    \item Interface tolerante a falhas: o sistema funciona mesmo que um barrel falhe;
    \item Persistência com ficheiros .ser e .txt;
    \item Crawlers paralelos podem ser iniciados manualmente.
\end{itemize}

\newpage

\section{Tolerância a Falhas e Confiabilidade}
\begin{itemize}
    \item A gateway tenta múltiplos barrels em ordem aleatória;
    \item Se um barrel falhar, o outro é tentado automaticamente;
    \item Os barrels são réplicas: ambos armazenam os mesmos dados;
    \item A fila central evita duplicação de indexação entre crawlers;
    \item Dados são recuperados ao reiniciar qualquer barrel;
    \item Cada componente é modular e independente.
\end{itemize}

\newpage

\section{Testes Realizados}
\begin{itemize}
    \item Adição de links via LinkAdder → verificação na fila e no WebCrawler;
    \item Verificação da indexação no terminal dos barrels;
    \item Pesquisa de termos e análise dos resultados;
    \item Consulta de backlinks funcionais;
    \item Simulação de falha de barrel → continuidade do sistema;
    \item Reinício de barrel → dados anteriores carregados corretamente;
    \item Múltiplos crawlers → sem duplicação de tarefas.
\end{itemize}

\newpage

\section{Divisão de Trabalho}
\begin{itemize}
    \item \textbf{Carlos Soares}: 
    \item \textbf{Miguel Machado}: 
    \item \textbf{Ambos}: 
\end{itemize}

\newpage

\section{Conclusão}
O sistema cumpre todos os requisitos funcionais e técnicos da Meta 1. Foi testado com sucesso em cenários de uso real e simulação de falhas. A comunicação via RMI, a modularidade dos componentes, e a facilidade de extensão fazem deste projeto uma base sólida para fases futuras.

\end{document}